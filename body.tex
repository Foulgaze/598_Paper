\section{Introduction}
The wave of a wireless signal is defined by three components: wavelength, wave speed, and frequency. In general, assuming the wave is traveling through air, wave speed will always be the speed of light. This leaves wavelength and frequency as the components of a wireless signal that  can control be controlled.  There exists a tradeoff between these values, as they are inversely proportional to each other. Wireless signals with a longer wavelength travel further but have lower frequencies and high frequency signals attenuate quickly. 

In wireless networking, this means that faster data transmission requires closer proximity to the source. This tradeoff can be seen in many walks of life. A common example is 5G cell towers using mmWave technology. While mmWave enables data transmission at speeds far exceeding its mid- and low-band counterparts, it suffers from extreme attenuation. Even a hand obstructing the line of sight to a cell tower can significantly degrade the signal (CITE). A similar distance tradeoff is seen in homes nationwide, where most consumer wireless routers support both 2.4GHz and 5GHz communication. The former of the two frequencies provides a larger range and deeper penetration but suffers in terms of speed when compared to the latter. 

This tradeoff between signal strength and distance puts consumers in a no win-situation: Should they place their wireless access point (AP) in a central location to ensure broader coverage at the expense of faster 5GHz speeds, or position it closer to their primary workspace to take advantage of the high-frequency signal? While this choice might seem trivial, the difference is significant, as 5GHz signals can offer up to 10 times the data transmission rate when compared to their 2.4GHz counterparts. 

The prior situation highlights a less-discussed issue with APs. Most consumers place their AP in one location and never move it again. This static setup contrasts sharply with the dynamic nature of human activity, as people move around their homes throughout the day. It is easy to imagine scenarios where someone is at the furthest point in their house from the AP, experiencing poor signal quality simply because the AP is unable to adapt to their changing location.

This paper addresses the static AP issue with two primary aims. First, it proposes a solution: a dynamic wireless access point. Second, it evaluates various routing methods and their ability to maintain signal strength for users on the network. A dynamic AP solves the tradeoff between signal strength and signal speed by eliminating it almost entirely. By moving the AP to the user, the user can experience the best of both worlds: consistently fast speeds without needing to worry about connection strength. At its core, a dynamic AP requires two key components: a means of movement and a method to connect to users. For our tests, we utilized the iRobot Create 3 as the movement platform and the ASUS RT-AC86U as the consumer-grade AP. In lieu of access to a large home, all tests were run within the Thomas M. Siebel Center for Computer Science on the University of Illinois Urbana-Champaign campus. 

\section{Background}
\subsection{Received Signal Strength Indicator}
Received signal strength indicator (RSSI) is a measurement of the power level of a received radio signal. For wireless networking, RSSI is used for determining the strength of a signal between a client and an AP. Signals will attenuate naturally as they are sent through some medium. This attenuation means that when an AP receives a signal, the power of the original sent signal will be reduced by some amount based on how far that signal traveled. This means that, in general, a signal that is being sent from a closer position to the AP will have a larger RSSI than a signal sent from further away. Due to multipath and environmental factors, this may not always be the case, but it is often true.

\subsection{iRobot Create 3}
The iRobot Create 3 is a product derived from the popular Roomba vacuum. Built on the same platform, it lacks vacuuming capabilities. For the purposes of this paper, the Robot is valued solely for its ability to move while carrying a load. Its movement options include rotating, moving forward or backward, and navigating based on an imaginary grid of points. The Robot does not feature built-in mapping or navigation capabilities. While external packages may provide advanced routing options, these are beyond the scope of this project.

\section{Related Work}
\subsection{RSSI Localization}
The most common form of localization is the Global Positioning System (GPS). While GPS is highly accurate outdoors due to its usage of satellites, it performs far worse indoors. GPS signals are prone to interference and are often too weak to penetrate indoor spaces.  This has led to research involving using RSSI as a method of indoor localization. In wireless communication, RSSI is used as a measure of distance between devices based on signal strength. With 3 APs measuring the RSSI of the same device, trilateration can be employed to estimate the true location of the device rather than just the distance \cite{rssiLocalization}.

Trilaterion works as follows: Each AP calculates its distance to the client device based on RSSI. Then each AP will draw a circle around itself, with the radius representing the measured distance. The intersection points of the circles drawn by each AP represent potential positions for the client. The only necessary information for this calculation is the RSSI and the real-world position of the APs. RSSI localization is effective when up close (less than 5 meters), but as the distance increases it becomes far less accurate. 

% \begin{figure}[tp]
% \centering
% \includegraphics{figures/mouse}
% \caption{\blindtext}
% \end{figure}


\section{System Design}
Although we know which devices and localization methods we are going to use, there are many ways to determine how the robot will read and react to different scenarios. In order to avoid bias and help focus on overall feasibility, we propose three diverse routing methods that can fully test the potential of consumer robotic access points.

\subsection{Roaming}
The roaming method is an intuitive approach to routing with an indicator like RSSI. The idea is to feel for the location of optimal RSSI in a home. The robot should start off by roaming around its starting point randomly. If the RSSI has increased a favorable amount while moving in a given direction, the robot should set its new starting point to its current location and begin roaming once again. The robot should eventually converge to a location with an RSSI above a desired threshold and wait until quality worsens, where it will then repeat the process.\par
Roaming has the benefit of, navigation aside, being dynamic to any environment with little user configuration. It could theoretically follow a user wherever they go and could be programmed to safely navigate the environment rather than move randomly, though this is not trivial.

\subsection{Spinning}
The goal of the spinning method is to narrow down the general direction of the client and intelligently move in that direction until a goal RSSI is achieved. Upon worsening, the method should wake and repeat. This method requires attaching two antennas to the top of the robot. We use this setup to get the location of the client by comparing the per antenna RSSI every X degrees (30 by default). If the ratio of \(\frac{a1}{a2}\) or \(\frac{a2}{a1}\) is larger than a threshold value (1.1 by default), we use that as our direction and move towards it for a distance before repeating all the steps. This should eventually converge to a location that can achieve some desired RSSI value.\par
Like Roaming, assuming navigation can be handled smoothly, this solution is dynamic to any environment and has no implied user configuration. This method, however, calculates before moving rather than constantly guessing. This should result in a higher movement efficiency but has a higher calculation overhead (as sampling RSSI and rotating the robot takes time).

\subsection{Save Points}
Save Points looks to take advantage of household habits by having an internal mapping of preferred spots around the house instead of being fully dynamic. The robot should linger at one of these positions if the RSSI is under a desired threshold, and should continue to recalculate it’s RSSI while it is physically idle. Once the connection worsens, the robot should switch to different preferred spots until the goal RSSI is seen again.\par
This method relies less on localization and has predictable movement, which may fit better in real world scenarios. Since it’s not dynamic, it has little calculation overhead, but requires the internet traffic to have some structure. It also requires the user to predict their own traffic in order to configure these points, which is what a smart robotic access point should be mitigating.


\section{Results/Validation}
In evaluating how well the proposed methods function, we will be measuring 2 factors. The first factor is RSSI. As mentioned in the introduction, the primary goal of this project is to ensure strong WiFi signal strength from any location within a home. If the RSSI between the client and the AP converges to a higher value, it indicates that the movement method effectively strengthens the client’s connection. The second factor measured is time to converge. If the RSSI does converge but it takes hours to do so, it will not be useful in real world situations. The movement methods should converge within minutes of the user moving to a new location, barring extreme movement by the client.  
\subsection{Roaming}

\begin{figure}[tp]
\centering
\includegraphics[scale=0.5]{figures/rssi_roaming}
\caption{RSSI of the Robot as it attempts to increase signal strength via the Roaming method. }
\end{figure}

While roaming may seem like an effective approach for lowering the RSSI quickly, it does quite the opposite. During our trials, we observed that the RSSI of the Robot did not consistently improve, often resulting in reduced connection quality. During our tests, we manipulated various factors in an effort to enhance roaming effectiveness, primarily adjusting the distance the robot traveled before re-averaging the RSSI. Despite changing the roaming parameters, the method itself proved ineffective. This can be seen in Figure 1 which is demonstrative of the results of our trials. The RSSI got worse over the 1.5 minute trial and had no clear convergence. 

When thinking about why roaming doesn’t work as a solution, we must revisit how RSSI works. RSSI performs best within a 5 meter radius, where changes in signal strength are most noticeable. In our tests, with the robot ~15 meters from the AP, the RSSI changes from the robot did not consistently register as smaller when moving towards the router. Our tests tried movement distances up to 3 meters without registering a consistent change in RSSI. A greater movement distance than this would likely be necessary for the Robot to register a consistent change. This is not really feasible, as the robot could potentially move very far away from the router, causing a signal drop, or it would be unable to move such a great distance easily, as it must navigate through a house and cannot move in straight lines.
\subsection{Spinning}
\begin{figure}[tp]
\centering
\includegraphics[scale=0.5]{figures/rssi_microwave}
\caption{RSSI of the Robot as it attempts to increase signal strength via the Spinning method.}
\end{figure}
Aligning the antennas such that the difference in signal strength is maximized does work in theory but falls flat in practice. As shown in Figure 2, there is no clear improvement in RSSI, and much like with roaming, the RSSI got worse over the majority of our trials. While the robot would sometimes move in the correct direction, this was never consistently the case. 

The issue with the spinning method is the antennas are not far enough apart to have a noticeable RSSI change between them. As stated earlier, RSSI changes are most noticeable within 5 meters, and when both antenna are ~15 meters away, RSSI is just not consistent enough for the difference in the antenna strengths to be significant. One could imagine a massive contraption that would place the antennas meters away from each other, but that would not be practical as it would 5x the floor profile of the robot. 

\subsection{Save Points}
\begin{figure}[tp]
\centering
\includegraphics[scale=0.5]{figures/rssi_save_points}
\caption{RSSI of the Robot as it attempts to increase signal strength via the Save points method.}
\end{figure}
Save points is the only algorithm we tested that consistently had successful results. During our tests, when we placed the client near a save point, the robot would consistently route to the nearest point and wait there until we moved the client to a new position. Figure 3 shows how long it took the robot to converge, but the timeline is not accurate due to the communication overhead between the robot and controller via Bluetooth. The robot converged in < 1 minute, ignoring communication, every time we placed the client in a position near a save point. This can be seen in Figure 3 where the RSSI plateaus around 200s and 650 seconds, which are both times the robot stopped as the RSSI was above the threshold for a strong connection. 

Save points works well because it offers a structured roaming approach. By moving between save points that are placed at larger distances, the problem of RSSI not changing is avoided, and there are no routing issues since the robot follows the points like a grid. There are two downsides to save points: The first is that someone has to manually determine the points, and the second is that the robot could still path out of range of the router, but this could be fixed with a more intelligent approach that prevents pathing in a direction once the RSSI becomes bad enough. 

\section{Conclusion}
\blindtext

