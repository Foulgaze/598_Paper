\section{Introduction}
The wave of a wireless signal is defined by three components: wavelength, wave speed, and frequency. In general, assuming the wave is traveling through air, wave speed will always be the speed of light. This leaves wavelength and frequency as the components of a wireless signal that  can control be controlled.  There exists a tradeoff between these values, as they are inversely proportional to each other. Wireless signals with a longer wavelength travel further but have lower frequencies and high frequency signals attenuate quickly. 

In wireless networking, this means that faster data transmission requires closer proximity to the source. This tradeoff can be seen in many walks of life. A common example is 5G cell towers using mmWave technology. While mmWave enables data transmission at speeds far exceeding its mid- and low-band counterparts, it suffers from extreme attenuation. Even a hand obstructing the line of sight to a cell tower can significantly degrade the signal (CITE). A similar distance tradeoff is seen in homes nationwide, where most consumer wireless routers support both 2.4GHz and 5GHz communication. The former of the two frequencies provides a larger range and deeper penetration but suffers in terms of speed when compared to the latter. 

This tradeoff between signal strength and distance puts consumers in a no win-situation: Should they place their wireless access point (AP) in a central location to ensure broader coverage at the expense of faster 5GHz speeds, or position it closer to their primary workspace to take advantage of the high-frequency signal? While this choice might seem trivial, the difference is significant, as 5GHz signals can offer up to 10 times the data transmission rate when compared to their 2.4GHz counterparts. 

The prior situation highlights a less-discussed issue with APs. Most consumers place their AP in one location and never move it again. This static setup contrasts sharply with the dynamic nature of human activity, as people move around their homes throughout the day. It is easy to imagine scenarios where someone is at the furthest point in their house from the AP, experiencing poor signal quality simply because the AP is unable to adapt to their changing location.

This paper addresses the static AP issue with two primary aims. First, it proposes a solution: a dynamic wireless access point. Second, it evaluates various routing methods and their ability to maintain signal strength for users on the network. A dynamic AP solves the tradeoff between signal strength and signal speed by eliminating it almost entirely. By moving the AP to the user, the user can experience the best of both worlds: consistently fast speeds without needing to worry about connection strength. At its core, a dynamic AP requires two key components: a means of movement and a method to connect to users. For our tests, we utilized the iRobot Create 3 as the movement platform and the ASUS RT-AC86U as the consumer-grade AP. In lieu of access to a large home, all tests were run within the Thomas M. Siebel Center for Computer Science on the University of Illinois Urbana-Champaign campus. 


\begin{figure}[tp]
\centering
\includegraphics{figures/mouse}
\caption{\blindtext}
\end{figure}

\section{Related Work}
\blindtext

And we need some citation here\cite{floyd1993random, stoica2001chord}

\Blindtext

\section{System Design}
Although we know which devices and localization methods we are going to use, there are many ways to determine how the robot will read and react to different scenarios. In order to avoid bias and help focus on overall feasibility, we propose three diverse routing methods that can fully test the potential of consumer robotic access points.

\subsection{Roaming}
The roaming method is an instinctive approach to routing with an indicator like RSSI. The idea is to feel for the location of optimal RSSI in a home. The robot should start off by roaming around its starting point randomly. If the RSSI has increased a favorable amount while moving in a given direction, the robot should set its new starting point to its current location and begin roaming once again. The robot should eventually converge to a location with an RSSI above a desired threshold and wait until quality worsens, where it will then repeat the process.\par
Roaming has the benefit of, navigation aside, being dynamic to any environment with little user configuration. It could theoretically follow a user wherever they go and could be programmed to safely navigate the environment rather than move randomly, though this is not trivial.

\subsection{Spinning}
The goal of the spinning method is to narrow down the general direction of the client and intelligently move in that direction until a goal RSSI is achieved. Upon worsening, the method should wake and repeat. This method requires attaching two antennas to the top of the robot. We use this setup to get the location of the client by comparing the per antenna RSSI every X degrees (30 by default). If the ratio of a1/a2 or a2/a1 is larger than a threshold value (1.1 by default), we use that as our direction and move towards it for a distance before repeating all the steps. This should eventually converge to a location that can achieve some desired RSSI value.\par
Like Roaming, assuming navigation can be handled smoothly, this solution is dynamic to any environment and has no implied user configuration. This method, however, calculates before moving rather than constantly guessing. This should result in a higher movement efficiency but has a higher calculation overhead (as sampling RSSI and rotating the robot takes time).

\subsection{Save Points}
Save Points looks to take advantage of household habits by having an internal mapping of preferred spots around the house instead of being fully dynamic. The robot should linger at one of these positions if the RSSI is under a desired threshold, and should continue to recalculate it’s RSSI while it is physically idle. Once the connection worsens, the robot should switch to different preferred spots until the goal RSSI is seen again.\par
This method relies less on localization and has predictable movement, which may fit better in real world scenarios. Since it’s not dynamic, it has little calculation overhead, but requires the internet traffic to have some structure. It also requires the user to predict their own traffic in order to configure these points, which is what a smart robotic access point should be mitigating.


\section{Results/Validation}
\Blindtext

\section{Conclusion}
\blindtext

