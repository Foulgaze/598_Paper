\documentclass[sigconf]{acmart}

\usepackage[english]{babel}
\usepackage{blindtext}

% Copyright
\renewcommand\footnotetextcopyrightpermission[1]{} % removes footnote with conference info
\setcopyright{none}
%\setcopyright{acmcopyright}
%\setcopyright{acmlicensed}
%\setcopyright{rightsretained}
%\setcopyright{usgov}
%\setcopyright{usgovmixed}
%\setcopyright{cagov}
%\setcopyright{cagovmixed}

\settopmatter{printacmref=false, printccs=false, printfolios=true}

% DOI
\acmDOI{}

% ISBN
\acmISBN{}

%Conference
%\acmConference[Submitted for review to SIGCOMM]{}
%\acmYear{2018}
%\copyrightyear{}

%% {} with no args suppresses printing of the price
\acmPrice{}


\begin{document}
\title{Exploring Feasibility of A Smart Robotic Wireless Access Point}

%\titlenote{Produces the permission block, and copyright information}
%\subtitle{Extended Abstract}

% \author{Paper \# XXX, XXX pages}
\author{Gabriel Righi \and Kendrit Tahiraj}
% \authornote{Note}
% \orcid{1234-5678-9012}
% \affiliation{%
%   \institution{Affiliation}
%   \streetaddress{Address}
%   \city{City} 
%   \state{State} 
%   \postcode{Zipcode}
% }
% \email{email@domain.com}

% The default list of authors is too long for headers}
\renewcommand{\shortauthors}{X.et al.}

\begin{abstract}
The traditional consumer WiFi setup involves a static access point somewhere in the center of their home. As the number of internet connected devices have increased in homes, the demand for larger access point coverage has followed. Companies have invested in “mesh” network solutions, which are systems with multiple access points in the home, however these can be expensive and unpredictable. In this paper, we explore the feasibility of a smart robotic wireless access point that can follow your home’s traffic wherever it goes. We evaluated three different implementations (Roaming, Spinning, Save Points) using consumer-level products and firmware. Using the Save Points implementation, we successfully increased the received signal strength of a dynamically positioned client by 10 dBm in less than two (adjusted) minutes. Our evaluation of these implementations produced both successes and failures that should be used as a starting point for future research involving smart robotic wireless access points.
\end{abstract}

\maketitle

\section{Introduction}
The wave of a wireless signal is defined by three components: wavelength, wave speed, and frequency. In general, assuming the wave is traveling through air, wave speed will always be the speed of light. This leaves wavelength and frequency as the components of a wireless signal that  can control be controlled.  There exists a tradeoff between these values, as they are inversely proportional to each other. Wireless signals with a longer wavelength travel further but have lower frequencies and high frequency signals attenuate quickly. 

In wireless networking, this means that faster data transmission requires closer proximity to the source. This tradeoff can be seen in many walks of life. A common example is 5G cell towers using mmWave technology. While mmWave enables data transmission at speeds far exceeding its mid- and low-band counterparts, it suffers from extreme attenuation. Even a hand obstructing the line of sight to a cell tower can significantly degrade the signal (CITE). A similar distance tradeoff is seen in homes nationwide, where most consumer wireless routers support both 2.4GHz and 5GHz communication. The former of the two frequencies provides a larger range and deeper penetration but suffers in terms of speed when compared to the latter. 

This tradeoff between signal strength and distance puts consumers in a no win-situation: Should they place their wireless access point (AP) in a central location to ensure broader coverage at the expense of faster 5GHz speeds, or position it closer to their primary workspace to take advantage of the high-frequency signal? While this choice might seem trivial, the difference is significant, as 5GHz signals can offer up to 10 times the data transmission rate when compared to their 2.4GHz counterparts. 

The prior situation highlights a less-discussed issue with APs. Most consumers place their AP in one location and never move it again. This static setup contrasts sharply with the dynamic nature of human activity, as people move around their homes throughout the day. It is easy to imagine scenarios where someone is at the furthest point in their house from the AP, experiencing poor signal quality simply because the AP is unable to adapt to their changing location.

This paper addresses the static AP issue with two primary aims. First, it proposes a solution: a dynamic wireless access point. Second, it evaluates various routing methods and their ability to maintain signal strength for users on the network. A dynamic AP solves the tradeoff between signal strength and signal speed by eliminating it almost entirely. By moving the AP to the user, the user can experience the best of both worlds: consistently fast speeds without needing to worry about connection strength. At its core, a dynamic AP requires two key components: a means of movement and a method to connect to users. For our tests, we utilized the iRobot Create 3 as the movement platform and the ASUS RT-AC86U as the consumer-grade AP. In lieu of access to a large home, all tests were run within the Thomas M. Siebel Center for Computer Science on the University of Illinois Urbana-Champaign campus. 


\begin{figure}[tp]
\centering
\includegraphics{figures/mouse}
\caption{\blindtext}
\end{figure}

\section{Related Work}
\blindtext

And we need some citation here\cite{floyd1993random, stoica2001chord}

\Blindtext

\section{System Design}

\subsection{The First Layer}
\Blindtext

\subsection{The Second Layer}
\Blindtext

\section{Evaluation}
\Blindtext

\section{Conclusion}
\blindtext



\bibliographystyle{ACM-Reference-Format}
\bibliography{reference}

\end{document}
